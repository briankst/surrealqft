\documentclass{article}
\usepackage{amsmath, amssymb, physics}
\usepackage{graphicx}
\usepackage{hyperref}

\begin{document}

\title{Surreal Quantum Field Theory: A Deterministic Framework for Quantum Mechanics and Gravity}
\author{Brian \and Grok 3 (xAI)}
\date{February 22, 2025}
\maketitle

\begin{abstract}
Surreal Quantum Field Theory (QFT) offers a deterministic unification of quantum mechanics (QM), quantum field theory, and general relativity (GR) using a subset of surreal numbers \(\mathbb{S}\), embedded into hyperreals \({}^*\mathbb{R}\). Infinitesimal tags (\(\epsilon_i\)) pre-set outcomes, aligning with a superdeterministic view while preserving measurement independence through statistical decoupling from experimental choices. The theory recovers Born statistics, resolves Bell inequalities locally, respects gauge and gravitational symmetries, and predicts subtle, falsifiable effects in the Cosmic Microwave Background (CMB), atomic spectroscopy, quantum optics, and gravitational waves, testable with next-generation experiments.
\end{abstract}

\section{Introduction}
Quantum mechanics (QM) and quantum field theory (QFT) have long grappled with foundational paradoxes, such as the measurement problem and non-locality, which challenge our understanding of reality. These issues are compounded when attempting to reconcile QM with general relativity (GR), where the probabilistic nature of quantum states clashes with the deterministic evolution of spacetime. *Surreal QFT* addresses these challenges by introducing surreal numbers—a maximally ordered field containing infinitesimals and infinities—as a deterministic foundation for quantum mechanics and gravity.

\subsection{Primer on Quantum Issues and Superdeterminism}
In standard QM, the wavefunction collapse upon measurement introduces an element of randomness, leading to philosophical debates about the nature of reality. Bell’s theorem further complicates matters by suggesting that any hidden variable theory must be non-local to reproduce quantum correlations, seemingly at odds with relativity’s prohibition on faster-than-light influences. Superdeterminism offers a loophole: if measurement choices and hidden variables are correlated through initial conditions, quantum correlations can be explained locally. However, this approach is often criticized as conspiratorial, implying a pre-arranged universe that undermines experimental freedom.

\subsection{Philosophical Rationale for Surreal Numbers}
Surreal numbers, introduced by Conway, provide a natural framework for embedding determinism into quantum mechanics. Their infinitesimals allow for the encoding of hidden variables with arbitrary precision, resolving quantum ambiguities without invoking randomness or non-locality. Unlike real numbers, which struggle to capture determinism in continuous systems, surreals offer a structured hierarchy of scales, making them uniquely suited for modeling sub-Planckian effects and unifying quantum and gravitational phenomena.

\section