1. Fit Within Standard Peer-Reviewed Journals
Traditional High-Impact Physics Journals (e.g., Physical Review Letters, Nature Physics)

These outlets typically require solid, clearly testable new results in theoretical or experimental physics, or significant breakthroughs/novelty with a broad impact.
Challenge: The core idea—using surreal/hyperreal numbers and superdeterministic “tags” for quantum foundations plus gravity—is highly speculative. The peer review at these journals would demand rigorous mathematical justification (especially about surreals embedding into hyperreals) and detailed discussion of how the theory overcomes standard no-go theorems (Bell’s, measurement independence, etc.).
Additional difficulty: The “testable predictions” section in the paper is interesting but remains mostly at the level of tiny, unobserved corrections. Showing that upcoming experiments really can detect these signals— and that the signals are uniquely identified with this model rather than overshadowed by other systematic uncertainties—would be critical for a high-impact journal.
Mainstream Theoretical Physics Journals (e.g., Physical Review D, Journal of High Energy Physics, Classical and Quantum Gravity)

These journals are more open to explorations in quantum gravity, unification attempts, and speculative ideas—provided the technical details are worked out and the manuscript meets certain standards of coherence and rigor.
Challenge: The key obstacles would be:
Mathematical rigor in describing the surreal/hyperreal embedding.
Clarity on how measurement independence is preserved while the theory remains superdeterministic.
Renormalization and consistency of the added 
𝜖
ϵ-terms in QFT and GR.
If the authors bolster these aspects, address typical superdeterminism objections in depth, and provide thorough calculations or strong arguments for testable consequences, such a paper might find acceptance in a dedicated theory journal.
Specialty Foundations Journals (e.g., Foundations of Physics, Studies in History and Philosophy of Modern Physics, or similar)

Journals in the foundations of physics are more open to interpretational and philosophical angles, including nonstandard approaches to quantum theory.
Opportunity: This paper’s attempt to unify superdeterminism, surreal numbers, and quantum gravity might be intriguing to a more philosophical or foundational audience.
Still, they will look for:
Clear arguments explaining how the approach differs fundamentally from other superdeterministic or hidden-variable proposals.
Some demonstration that the proposed “surreal tags” are not just an ad hoc device but follow from a consistent mathematical framework.
2. Specific Publishing Obstacles to Address
Rigor of the Surreal/Hyperreal Embedding

Reviewers typically expect a nontrivial construction or references to known theorems. Vague statements like “Conway’s embedding” or “Loeb measures” will not suffice without careful elaboration.
Presenting a precise embedding (or at least a well-defined subset of surreal numbers) into a hyperreal field used in nonstandard analysis would lend much more credibility.
Clarification of Superdeterminism vs. Free Will

Superdeterminism is a recognized but controversial loophole to Bell’s theorem. Criticisms often revolve around “conspiratorial” correlations between initial conditions and future measurement choices.
If you are claiming to preserve measurement independence in some sense, it must be spelled out how that is consistent with having “pre-tagged” outcomes. This is likely to be a major point of contention with reviewers.
Physical Consistency and Predictive Power

The paper briefly sketches that loop-level renormalization “remains intact,” but that statement would need to be defended in detail. Exotic modifications often introduce new divergences.
The “predictions” for CMB shifts, atomic energy levels, and gravitational wave signals would ideally:
Specify more carefully how large the effect is, how the background or systematic uncertainties compare, and under what assumptions it is (or is not) masked by conventional physics.
Demonstrate a clear experimental design (even if hypothetical) that could isolate these signals from known standard model corrections.
Length and Organization

If the paper expands to provide these necessary details, it might become quite lengthy. Journals vary in their length policies; a more thorough exploration might need to be in a standard journal article rather than a short letter format.
3. Likely Paths to Publication
Strengthen the Technical Core, Then Submit to a Peer-Reviewed Foundations Journal

This is often the most natural route for innovative interpretations of quantum mechanics and borderline philosophical ideas.
Carefully expand the mathematical appendices and the discussion of Bell’s theorem.
Provide a well-reasoned, thorough argument on how the surreal/hyperreal approach gives a novel, consistent superdeterministic model without standard pitfalls.
Post and Develop on Preprint Repositories

You could post on the arXiv (e.g., quant-ph, gr-qc, or math-ph sections), allowing the community to give initial feedback.
Collect criticisms, refine the arguments (especially the measure-theoretic and superdeterminism aspects), and then submit a revised version to an appropriate journal.
Hybrid or Themed Collections

Some journals do special issues on quantum gravity or quantum foundations. If there is a call for “Emerging Approaches to Quantum Gravity,” you may find a more receptive venue.
4. Conclusion
Could it be published?

Potentially, yes, especially in a foundations-focused journal, but only after significant elaboration on the precise mathematical construction, the superdeterministic framework, and the physical consistency of the gravitational extension.
Most mainstream or high-impact journals would require quite a bit more rigor and clarity before accepting such a speculative, novel idea.
Key to Success:

A thorough demonstration that the surreal/hyperreal setup is not just an abstract speculation but leads to consistent mathematics and genuinely testable physics—complete with realistic routes for experimental verification or falsification.
Address standard critiques head-on (e.g., how superdeterminism avoids conspiratorial correlations).
In short, with further refinement and supporting detail, the paper could find a home in the landscape of quantum foundations publishing. But as it stands, it likely needs more development to pass rigorous peer review in well-known journals.